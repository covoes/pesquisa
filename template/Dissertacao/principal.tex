% ---------------------------------------------------------------------------
% ---------------------------------------------------------------------------
% Junção de templates encontrados no Overleaf
% facilitando a vida do aluno de pós da UFABC
% Modelo LaTex para preparação do documento final de Dissertação de Mestrado
% ninguém do programa de pós da informação validou se presta
% ---------------------------------------------------------------------------
% ---------------------------------------------------------------------------

\documentclass[
	% -- opções da classe memoir --
	12pt,					% tamanho da fonte
	openright,				% capítulos começam em pág ímpar (insere página vazia caso preciso)
	twoside,					% para impressão em verso e anverso. Oposto a oneside
	a4paper,					% tamanho do papel.
	% -- opções da classe abntex2 --
	%chapter=TITLE,			% títulos de capítulos convertidos em letras maiúsculas
	%section=TITLE,			% títulos de seções convertidos em letras maiúsculas
	%subsection=TITLE,		% títulos de subseções convertidos em letras maiúsculas
	%subsubsection=TITLE,	% títulos de subsubseções convertidos em letras maiúsculas
	% -- opções do pacote babel --
	english,					% idioma adicional para hifenização
	%french,					% idioma adicional para hifenização
	%spanish,				% idioma adicional para hifenização
	brazil					% o último idioma é o principal do documento
	]{abntex2}

% ---------------------
% Pacotes OBRIGATÓRIOS
% ---------------------
\usepackage{lmodern}				% Usa a fonte Latin Modern
\usepackage[T1]{fontenc}			% Selecao de codigos de fonte.
\usepackage[utf8]{inputenc}		% Codificacao do documento (conversão automática dos acentos)
\usepackage{lastpage}			% Usado pela Ficha catalográfica
\usepackage{indentfirst}			% Indenta o primeiro parágrafo de cada seção.
\usepackage{color}				% Controle das cores
\usepackage{graphicx,graphicx}	% Inclusão de gráficos
\usepackage{epsfig,subfig}		% Inclusão de figuras
\usepackage{microtype} 			% Melhorias de justificação
\usepackage{amsmath}
\usepackage{amsfonts}
\usepackage{amssymb}
% ---------------------

% ---------------------
% Pacotes ADICIONAIS
% ---------------------
\usepackage{lipsum}						% Geração de dummy text
\usepackage{amsmath,amssymb,mathrsfs}	% Comandos matemáticos avançados
\usepackage{setspace}  					% Para permitir espaçamento simples, 1 1/2 e duplo
\usepackage{verbatim}					% Para poder usar o ambiente "comment"
\usepackage{tabularx} 					% Para poder ter tabelas com colunas de largura auto-ajustável
\usepackage{afterpage} 					% Para executar um comando depois do fim da página corrente
\usepackage{url} 						% Para formatar URLs (endereços da Web)
% ---------------------

% ---------------------
% Pacotes de CITAÇÕES
% ---------------------
\usepackage[brazilian,hyperpageref]{backref}	% Paginas com as citações na bibl
\usepackage[alf]{abntex2cite}				% Citações padrão ABNT (alfa)
%\usepackage[num]{abntex2cite}				% Citações padrão ABNT (numericas)
% ---------------------

% Configurações de CITAÇÕES para abntex2
\include{extras/conf_citacoes}

% Inclusão de dados para CAPA e FOLHA DE ROSTO (título, autor, orientador, etc.)
% ---
% Informações de dados para CAPA e FOLHA DE ROSTO
% ---
\titulo{Seu título aqui}
\autor{Seu nome Aqui}
\local{Santo André - SP}
\data{2020}
\orientador{Prof. Dr. Thiago Ferreira Covões}
\coorientador{}
\instituicao{%
  Universidade Federal do ABC -- UFABC
  \par
  Centro de Matemática, Computação e Cognição 
  \par
  Programa de Pós-Graduação em Ciência da Computação}
\tipotrabalho{Dissertação (Mestrado)}
% O preambulo deve conter o tipo do trabalho, o objetivo,
% o nome da instituição e a área de concentração
\preambulo{\textbf{Dissertação de Mestrado} apresentada ao Programa de Pós-Graduação da Universidade Federal do ABC como requisito parcial para obtenção do grau de Mestre em Ciência da Computação}
% ---


% Inclui Configurações de aparência do PDF Final
\include{extras/conf_pdf}

% O tamanho da identação do parágrafo é dado por:
\setlength{\parindent}{1.3cm}

% Controle do espaçamento entre um parágrafo e outro:
\setlength{\parskip}{0.2cm}  % tente também \onelineskip

% ---------------------
% Compila o indice
% ---------------------
\makeindex
% ---------------------

%%%%%%%%%%%%%%%%%%%%%%%%%%%
%%  INICIO DO DOCUMENTO  %%
%%%%%%%%%%%%%%%%%%%%%%%%%%%
\begin{document}

% Retira espaço extra obsoleto entre as frases.
\frenchspacing

% ----------------------------------------------------------
% ELEMENTOS PRÉ-TEXTUAIS (Capa, Resumo, Abstract, etc.)
% ----------------------------------------------------------
\pretextual

% Capa


\thispagestyle{empty}
\pagenumbering{roman}
\DeclareGraphicsExtensions{.jpg,.pdf,.svg,.mps,.png} % Tipos de arquivos de imagem que ser�o utilizados

%%%%%%%%%%%%%%% CAPA %%%%%%%%%%%%%%%%%%%%%%
\begin{center}
{\LARGE
\vspace*{0.1cm}
\textbf{Seu nome aqui}}\\


\vspace{5.0cm}

{\LARGE
\textbf{Seu t�tulo aqui}}

\end{center}


\vspace{4.0cm}

\apresentacao{Projeto de Gradua��o em Computa��o submetido �
Universidade Federal do ABC para a
obten��o dos cr�ditos na disciplina Projeto de Gradua��o em Computa��o III do curso de Ci�ncia
da Computa��o}

\vspace{2.0cm}
    {\large
    \noindent
    \textbf{Orientador:} Prof  Dr Thiago Ferreira Cov�es}\\



\vspace*{\stretch{1}} %%posicionamento para colocar data


\begin{center}

{\large
Universidade Federal do ABC\\
\hoje}
\end{center}

\thispagestyle{empty}

\newpage


% Folha de rosto (o * indica que haverá a ficha bibliográfica)
\imprimirfolhaderosto*

% Imprimir Ficha Catalografica
\include{pretextual/catalografica}

% Inserir Folha de Aprovação
\include{pretextual/assinaturas}

% Dedicatória
% ---
% Dedicatória
% ---
\begin{dedicatoria}
   \vspace*{\fill}
   \centering
   \noindent
   \textit{À ciência que facilita o acesso à informação} \vspace*{\fill}
\end{dedicatoria}
% ---

% Agradecimentos
% ---
% Agradecimentos
% ---
\begin{agradecimentos}

Voltarei com tempo à esta seção.

\end{agradecimentos}
%% ---

% Epígrafe
\include{pretextual/epigrafe}

% Resumo e Abstract
% ---
% RESUMOS
% ---

% RESUMO em português
\setlength{\absparsep}{18pt} % ajusta o espaçamento dos parágrafos do resumo
\begin{resumo}

 \textbf{Palavras-chaves}: palavra1, palavra2..
\end{resumo}

% ABSTRACT in english
\begin{resumo}[Abstract]
 \begin{otherlanguage*}{english}

   \vspace{\onelineskip}
 
   \noindent 
   \textbf{Keywords}: word 1, word 2..
 \end{otherlanguage*}
\end{resumo}


% Lista de ilustrações
\pdfbookmark[0]{\listfigurename}{lof}
\listoffigures*
\cleardoublepage

% Lista de tabelas
\pdfbookmark[0]{\listtablename}{lot}
\listoftables*
\cleardoublepage

% Lista de abreviaturas e siglas
\begin{siglas}
  \item[ABNT] Associação Brasileira de Normas Técnicas
  \item[abnTeX] Normas para TeX
\end{siglas}

% Lista de símbolos
\begin{simbolos}
  \item[$ \Gamma $] Letra grega Gama
  \item[$ \Lambda $] Lambda
  \item[$ \zeta $] Letra grega minúscula zeta
  \item[$ \in $] Pertence
\end{simbolos}

% Inserir o SUMÁRIO
\pdfbookmark[0]{\contentsname}{toc}
\tableofcontents*
\cleardoublepage

% ----------------------------------------------------------
% ELEMENTOS TEXTUAIS (Capítulos)
% ----------------------------------------------------------
\textual
% Elementos textuais com numeração arábica
\pagenumbering{arabic}
% Reinicia a contagem do número de páginas
\setcounter{page}{1}

% Inclui cada capitulo da Dissertação
\include{capitulos/introducao}
\chapter{Fundamentação Teórica}\label{cap:fundamentacao_teorica}



\chapter[Trabalhos Relacionados]{Trabalhos Relacionados}



\include{capitulos/metodologia}


% ----------------------------------------------------------
% ELEMENTOS PÓS-TEXTUAIS (Referências, Glossário, Apêndices)
% ----------------------------------------------------------
\postextual

% Referências bibliográficas
\bibliography{bibliografia}

% Glossário (Consulte o manual)
%\glossary

% Apêndices
\include{postextual/apendices}

% Anexos
\include{postextual/anexos}

% Índice remissivo (Consultar manual)
%\phantompart
%\printindex

\end{document}
